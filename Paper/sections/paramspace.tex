\documentclass[../Paper.tex]{subfiles}
\begin{document}
  In order to determine the smallest inputs into the model which would result in
  an epidemic, it is necessary to visualize the hyperparameter space via grid search.
  \\
  Let our two parameter inputs be defined as:
  \begin{equation}
    x \in \{1, 2, \cdots, n \}
  \end{equation}
  and,
  \begin{equation}
    y \in \{1, 2, \cdots, n \}
  \end{equation}
  where: $x =$ number of new cases of measles being added to the model, and
  $y =$ number of ``insertion events'' which occur during one model simulation.\\
  \\
  The cost function, $f(x, y)$, provides scalar output of the maximum outbreak length
  when the model is evaluated with $x$ and $y$ as inputs.\\
  \\
  The three dimensional surface of the hyperparameter space can be found by evaluating $f(x,y)$
  with all reasonable combinations of $x$ and $y$. The three values can then be
  expressed as cartesian coordinates and plotted. Importantly, this will help to reveal
  local maxima, minimizing for the input values. A random search or
  genetic algorithm methods could also have been used to this end, but due to the
  suspected spread of local maxima and minima, a whole grid search of the
  hyperparameter space provides more robust results, depite the time penalty.
  \\
  % Subfile for plot is not necessary, but it makes the file look better
  \subfile{sections/paramspacegraph}
  \clearpage
\end{document}
