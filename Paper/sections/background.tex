\documentclass[../Paper.tex]{subfiles}
\begin{document}
\raggedright
Measles is one of 25 vaccine preventable illnesses identified by the WHO as a
global eradication target in their Global Vaccine Action Plan. While elimination
efforts in some areas of the globe have seen great success, Europe still bears a
disproportionate burden of the disease. This is due, at least in part, to low
vaccination coverage in several nations\cite{world_health_organization_2017}.

While data on effective vaccination rates vary based on national surveillance
capacity, regional efforts, like the European Sero-Epidemiology Network (ESEN),
have provided more robust measurements ranging over a decade. Their findings
reveal that all nations with an endemic spread of the disease fall short of the
herd immunity threshold (around 96\%)\cite{andrews_tischer_siedler_pebody_2008}.
This shortfall reflects the recent trend of hesitancy on the part of parents to
immunize their child with both doses of the MMR vaccine required for full coverage.

Beyond the immediate effects of endemic measles transmission within a country,
it is reasonable to question whether these nations also put the elimination efforts
of their neighbors at risk. It is this question that this project seeks to answer,
particularly as more nations start to reach their elimination targets and are declared
``measles eliminated'' by the European divsion of the WHO\cite{sniadack_crowcroft_durrheim_rota_2017}.

\clearpage
\end{document}
