\documentclass[../Paper.tex]{subfiles}
\begin{document}
Model stochasticity is accomplished through the R package, adaptivetau\cite{johnson_2016}.
The deterministic differential equations are supplied as a rate function evaluated
by adaptivetau. Each transition defined by the rate function (movement between compartments)
is stepped using a process called "explicit tau-leaping"\cite{yang_gillespie_petzold_2007}.
\par
This process provides an approximation of the output expected from Gillespie algorithm by maximizing
the time step between data points while minimizing the rate of change of the transition.
The resulting output supplies many data points where the rate of change in a transition
is high (in this model, the outbreak curve) and sparse data points where the rate
of change is slow. This serves a dual purpose:
\par
\begin{enumerate}
  \item[$\bullet$]{To porovide vastly superior performance to the Gillespie algorithm}
  \item[$\bullet$]{To introduce model stochasticity by using a poisson-distributed
                   random walk variable to define transition advancement}
\end{enumerate}

\clearpage
\end{document}
