\documentclass[../Paper.tex]{subfiles}
\begin{document}
\justifying
Population demographic data for the project was acquired from the United Nations
Department of Economic and Social Affairs, Population Division (UN DESA). Measles
seroprevalence data was made available by findings from ESEN2 presented to the World
Health Organization.

As previously mentioned, the model will have two age compartments in an effort to
reduce the number of dimensions. However, the WHO report provides seroprevalence data
in several age categories, and the appropriate proportion of measles seronegativity
in each age group must be extrapolated. Additionally, the data in the
seroepidemiology report were not all collected in the same year.
Several countries, such as Sweden, have not submitted updated data in over 20 years,
therefore some liberty was taken in best estimating these population values.
Where possible, sensitivity analyses of the parameters were conducted to
quantify the impact on model results.

To extrapolate values for in the model, seroprevalence values were direct standardized
using the most recent population age demographics for each country. Two summary
measures were created by summating the weighted rates for each age range
contained in the model compartments. This is described here by the following
equation:
  \begin{equation}
    P_{[i,n]} =\sum_{i}^{n}W_{i}\cdot \widehat{p_{i}}
  \end{equation}
  where:\\
  $P_{[i,n]} =$ measles seroprevalence from ages $i$ to $n$, \\
  $W_{i} =$ weight of age group $i$, calculated as the quotient of the population
  of age group $i$ divided by the population in groups $i$ to $n$, and \\
  $\widehat{p_{i}} =$ measles seroprevalence in group $i$

The resulting seroprevalence values are then multiplied by their respective
populations to generate the proportion susceptible and immune in the two age
compartments.
\clearpage
\end{document}
